\documentclass{exam}
\usepackage{graphicx}
\graphicspath{ {images/} }
\usepackage{amsmath}
\usepackage{array}
\usepackage{tabularx}
\usepackage[export]{adjustbox}
\usepackage{pythonhighlight}
\printanswers
\renewcommand{\solutiontitle}{\noindent\textbf{Resposta:}\par\noindent}
\begin{document}
\begin{center}
\fbox{\fbox{\parbox{5.5in}{\centering
Universidade Federal de Santa Catarina \\ [1ex]
Departamento de Informática e Estatística \\ [1ex]
INE 5426 - Construção de Compiladores \\ [1ex]
Relatório \\ [1ex] 

}}}
\end{center}

\begin{flushleft}
\vspace{5mm}
\makebox[0.542\linewidth]{Nome: Fabio Oliveira de Abreu (18100529)}
\makebox[0.595\linewidth]{Nome: Bruno Duarte Barreto Borges (18100519)}
\makebox[0.52\linewidth]{Nome: Erik Kazuo Sugawara (18100528)}

\vspace{5mm}
\makebox[0.5\linewidth]{Professor: Alvaro Junio Pereira Franco}
\end{flushleft}

% ------  Questions ------
\vspace{5mm}
\begin{questions}
    \question Identificação dos tokens.
        \begin{solution}
            A identificação inicial dos tokens foi retirada da grámatica CC-2021-2,
            onde, para facilitar a organização dentro do código, separamos cada token em cinco grupos:
            palavras reservadas, operadores, símbolos especiais, constantes e identificadores. 
            Seguem alguns exemplos de tokens com a sua expressão regular.
            \begin{python}
t_ASSIGN = r'\='
t_GT = r'\>'
t_LT = r'\<'
t_EQ = r'\=='
t_LE = r'\<='
t_GE = r'\>='
t_NEQ = r'\!='
t_PLUS = r'\+'
t_MULTIPLY = r'\*'
t_DIVIDE = r'\/'
t_REM = r'\%'
            \end{python}

        \end{solution}
    \question Produção das definições regulares para cada token.
        \begin{solution}

            Para produção das definições regulares, foram utilizadas expressões regulares.
            Segue um exemplo, da definição regular do token "FLOAT" utilizando a ferramenta PLY.
            \begin{python}
def t_float_constant(t):
    r'[+-]?\d+\.\d+'
    t.value = float(t.value)
    return t
            \end{python}    
        \end{solution}

    \question Construção dos diagramas de transição para cada token
        \begin{solution}
            Os diagramas de transição dos tokens foram feitos em uma tabela xls, que se localiza na pasta xxxx.
        \end{solution}

    \question Descrição de uma tabela de símbolos (como foi implementada),
    quais são os símbolos armazenados na tabela e quais são os atributos dos
    símbolos escolhidos para armazenar na tabela
        \begin{solution}
            A descrição da tabela de símbolos foi feita através dos tokens que eram retornados pelo analisador léxico,
            onde, dado uma entrada, a ferramenta identificava seu respectivo token. Com todos tokens identificados, é
            utilizado o método "print\_table" que imprime a tabela de forma elegante no próprio terminal. Os atributos
            que foram escolhidos para serem mostrados na tabela foram: Token, Valor, Linha e Coluna.
            \begin{python}
def print_table(lexer):
    pattern = "{:^25} | {:^20} | {:^5} | {:^5}"
    print("\033[4m" + pattern.format("TOKEN", "VALUE", "L", "C") + "\033[0m")
    while True:
        tok = lexer.token()
        if not tok:
            break
        print(pattern.format(tok.type, tok.value, tok.lineno, find_column(tok))) 

    for e in errors:
        print(e)
            \end{python}
        \end{solution}
    \question Se não usou ferramenta, uma descrição da implementação do analisador
     léxico (Usou diagramas de transição? Quais? Quantos? Se não usou
    diagramas de transição, então o que foi usado?)
        \begin{solution}
            Foi utilizado a ferramenta PLY (Python Lex-Yacc).
        \end{solution}

    \question Se usou ferramenta, uma descrição detalhada da entrada exigida
    pela ferramenta e da saída dada por ela. É necessário haver exemplos
    pequenos da entrada e da saída gerada pela ferramenta com essa entrada.
        \begin{solution}
            Para a criação do analisador léxico, foi utilizada o PLY (Python Lex-Yacc),
            uma implementação das ferramentas lex e yacc para python. Nela, podemos
            criar uma lista de tokens a serem identificados pelo analisador, onde devem
            ser armazenados em uma lista chamada "tokens". As váriaveis
            e funções que se iniciam pelos caracteres "\_t" são identificadas como um token
            pelo interpretador, onde podemos definir suas respectivas expressões regulares.
        \end{solution}

\end{questions}
\end{document}