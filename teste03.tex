\documentclass{exam}
\usepackage{graphicx}
\graphicspath{ {images/} }
\usepackage{amsmath}
\usepackage{array}
\usepackage{tabularx}
\usepackage[export]{adjustbox}
\printanswers
\renewcommand{\solutiontitle}{\noindent\textbf{Resposta:}\par\noindent}
\begin{document}
\begin{center}
\fbox{\fbox{\parbox{5.5in}{\centering
Universidade Federal de Santa Catarina \\ [1ex]
Departamento de Informática e Estatística \\ [1ex]
INE 5405 - Probabilidade e Estatística \\ [1ex]
Teste 03 \\ [1ex] 

}}}
\end{center}

\begin{flushleft}
\vspace{5mm}
\makebox[0.51\textwidth]{Nome: Erik Kazuo Sugawara (18100528)}

\vspace{5mm}
\makebox[0.7\textwidth]{Professor: Jose Francisco Danilo de Guadalupe Corrêa Fletes}
\end{flushleft}

% ------  Questions ------
\vspace{5mm}
\begin{questions}
%    680,677,649,681,652,717,704,683,662,681,669,669,675,679,720,727,652,723,644,715,
%    719,700,701,691,660,653,664,710,683,665,699,660,676,683,695,638,702,680,695,670,
%    678,690,683,705,655,661,665,684,674,675,670,681,735,742,693,680,688,674,671,688
% Max:  742
% Min:  638
% Question 01
\question (Vale 2,0): Apresente os dados em ramo-e-folhas (Exemplo à figura 3.11)
 e comente importantes características que Você observa (quanto ao padrão de
 comportamento dos dados). 
\begin{parts}
    \part Ramo-e-folhas
        \begin{solution}
            Em minha observação, considerando o padrão de comportamento dos dados,
            a velocidade (em MHz) dos dispositivos se concentram na faixa de [660, 699] MHz.
            Além disso, podemos notar casos atípicos de velocidades como 638 MHz e 742 MHz, que fogem do padrão.
            Não faria sentido, portanto, utilizar esses casos extremos para determinar o preço do produto.
            Sendo assim, o fabricante deve levar em consideração os dispositivos que estão entre [660, 699]
            para determinar o seu preço. 
            \vspace{5mm}
            \begin{center}
                \begin{tabular}{|p{1cm}|p{3.5cm}|p{2cm}|}
                \hline  
                Ramo & Folha(s) & Frequência \\ [1ex] 
                \hline
                63 & 8 & 1 \\ [1ex] 
                \hline
                64 & 4,9 & 2 \\ [1ex] 
                \hline
                65 & 2,2,3,5 & 4 \\ [1ex] 
                \hline
                66 & 0,0,1,2,4,5,5,9,9 & 9 \\ [1ex] 
                \hline
                67 & 0,0,1,4,4,5,5,6,7,8,9 & 11 \\ [1ex] 
                \hline
                68 & 0,0,0,1,1,1,3,3,3,3,4,8,8 & 13 \\ [1ex]
                \hline
                69 & 0,1,3,5,5,9 & 6 \\ [1ex]
                \hline
                70 & 0,1,2,4,5 & 5 \\ [1ex]
                \hline
                71 & 0,5,7,9 & 4 \\ [1ex]
                \hline
                72 & 0,3,7 & 3 \\ [1ex]
                \hline
                73 & 5 & 1 \\ [1ex]
                \hline
                74 & 2 & 1 \\ [1ex]
                \hline

            \end{tabular}  
            \end{center}
        \end{solution} 
    \vspace{5mm}
    \part Qual a percentagem dos dispositivos que tem velocidade que excede 700 MHz?
        \begin{solution}
            A percentagem de dispositivos que excedem 700 Mhz é de 21,66\%.
        \end{solution} 
\end{parts}

% Question 02
\question (Vale 5,0): Construa para os dados dos 60 dispositivos a
 Tabela/Distribuição de Frequências (Três primeiras colunas da tabela abaixo): 
        \begin{solution}
              \begin{center}
                \begin{tabular}{|p{2cm}|p{2.5cm}|p{2cm}|p{1cm}|p{1cm}|p{2.5cm}|}
                \hline
                Classes & Contagem & Frequência \\ [1ex] 
                \hline
                638 $\vdash$ 653 & $|||||$ & 5 \\ 
                \hline
                653 $\vdash$ 668 & $|||||$ $||||$ & 9  \\ 
                \hline
                668 $\vdash$ 683 & $|||||$ $|||||$ $|||||$ $||||$ & 19 \\ 
                \hline
                683 $\vdash$ 698 & $|||||$ $|||||$ $||$ & 12 \\
                \hline
                698 $\vdash$ 713 & $|||||$ $||$ & 7 \\
                \hline
                713 $\vdash$ 728 & $|||||$ $|$ & 6 \\
                \hline
                728 $\vdash$ 743 & $|||||$ & 2 \\
                \hline

            \end{tabular}  
            \end{center}
        \end{solution} 

\vspace{5mm}

\question (Vale 1,0): Preencha a tabela anterior com os valores dos Pontos Médios 
de Classe, assim como com as frequências relativas - fi (ou percentual de 
observações - \%) e as frequências relativas acumuladas (ou percentual acumulado - Fj)).
Aplique as expressões à página 62: (3.1) e (3.2).
        \begin{solution}
            \begin{center}
                \begin{tabular}{|p{2cm}|p{2.5cm}|p{2cm}|p{1cm}|p{1.2cm}|p{2.5cm}|}
                \hline
                Classes & Contagem & Frequência & Ponto Médio & \% & \% Acumulada \\ [1ex] 
                \hline
                638 $\vdash$ 653 & $|||||$ & 5 & 645,5 & 8,33 \% & 8,33 \% \\ 
                \hline
                653 $\vdash$ 668 & $|||||$ $||||$ & 9 & 660,5 & 15 \% & 23,33 \% \\ 
                \hline
                668 $\vdash$ 683 & $|||||$ $|||||$ $|||||$ $||||$ & 19 & 675,5 & 31,66 \% & 55 \% \\ 
                \hline
                683 $\vdash$ 698 & $|||||$ $|||||$ $||$ & 12 & 690,5 & 20 \% & 75 \% \\
                \hline
                698 $\vdash$ 713 & $|||||$ $||$ & 7 & 705,5 & 11,66 \% & 86,65 \% \\
                \hline
                713 $\vdash$ 728 & $|||||$ $|$ & 6 & 720,5 & 10 \% & 96,65 \% \\
                \hline
                728 $\vdash$ 743 & $|||||$ & 2 & 735,5 & 3,33 \% & 100 \% \\
                \hline

            \end{tabular}  
            \end{center}
        \end{solution} 
\vspace{10cm}
\question (Vale 2,0): Construa a Distribuição de Frequências Acumuladas 
(exemplo figura 3.8).
        \begin{center}
        \includegraphics[width=17cm,height=20cm, keepaspectratio]{dist_freq4.png}
        \end{center}
        \begin{parts}
            \part Com base na expressão (3.3) da página 62, obtenha os valores aproximados da  variável em análise ( X ), considerando o gráfico obtido da F(x), de forma que:
            \begin{solution}
            \begin{equation}
                f(x) = P(X \leq Q_1) = 0.25 (25\%) \Longrightarrow \textrm{Valor de X} = Q_1 \approx 655
            \end{equation}

            \begin{equation}
                f(x) = P(X \leq Q_2) = 0.50 (50\%) \Longrightarrow \textrm{Valor de X} = Q_2 \approx 665
            \end{equation}

            \begin{equation}
                f(x) = P(X \leq Q_3) = 0.75 (75\%) \Longrightarrow \textrm{Valor de X} = Q_3 \approx 680
            \end{equation}
                
            \end{solution} 
        \end{parts}

 
\end{questions}
\end{document}